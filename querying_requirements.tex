\section{Querying requirements}



\noindent Here we present a list of queries that one will be expected 
to perform on the DB based on the REST API of OpenTox.

\noindent General remarks:
\begin{enumerate}
 \item Consistency on delete is not an important feature of Jaqpot.
 \item In a document database we don't have foreign keys. Their counterpart are 
 links to external objects (\textit{e.g.}, a URI, or the ID of some other object
 in the database). However, these links are weak and the advantage this offers
 is that we can combine resources from different servers. For the first time, we'll
 be able to have datasets with substances from different services!
\end{enumerate}


\subsection{Tasks and Error Reports}

\noindent Read operations:
\begin{enumerate}
 \item Get task representation by ID
 \item Get a list of task IDs:
    \begin{enumerate}
    \item from a given user
    \item with a given status (\textit{e.g.}, running)
    \end{enumerate}
 \item List all task IDs (paginated)
\end{enumerate}

\noindent Write operations:
\begin{enumerate}
 \item Write new task as \texttt{QUEUED} (default)
 \item Update the meta-data and/or status of a task
\end{enumerate}

\noindent Remarks:
\begin{enumerate}
 \item 	Here we have the most write and read intensive DB 
	operations. Tasks can well be stored in an independent 
	in-memory DB such as Redis. However, this would mean 
	that when we restart the server, all tasks will be lost. 
	Not sure this is the best choice. We can still store the 
	tasks in a document DB (e.g., CouchBase).
 \item 	In case of nested ErrorReport entities, children reports are 
        only fetched with their parents.
\end{enumerate}



\subsection{Model}

\noindent Read operations:
\begin{enumerate}
 \item Retrieve a model: Get model representation (metadata only -- 
       no need to load the actual model) given the ID of the model.
 \item Fetch only the actual model (binary object) but not the metadata. 
 \item List all models of a user
 \item List all models of an algorithm 
 \item List all models for a given keyword (facilitate searching in meta-data)
 \item Combinations of the two above (by user and algorithm)
 \item List all model IDs (paginated)
\end{enumerate}


\noindent Write operations:
\begin{enumerate}
 \item Register new model (all info)
 \item Update model to link to a new BibTeX
 \item Update the meta-data of a model (\textit{e.g.}, new title)
 \item Delete or disable model 
\end{enumerate}


\noindent Remarks:
\begin{enumerate}
 \item In models as well as in other resources, when retrieving data from the 
       DB there will never be the need to fetch the user data (only the ID 
       of the user is OK).
 \item Notice that the actual model (binary) and its metadata are never loaded simultaneously.
 \item In the OpenTox API one finds the method \texttt{GET /model/{id}/independent} 
       (and related methods), but there is not need to create very specialized 
       queries to serve these cases. If the model is stored as a document in the DB, 
       it can be retrieved as a whole and then we can pick only the information we are interested in.
\end{enumerate}





\subsection{User}

\noindent Read operations:
\begin{enumerate}
 \item List all user IDs
 \item Get the representation of a user by ID
 \item Get user's quota (part of the user representation actually)
\end{enumerate}


\noindent Write operations:
\begin{enumerate}
  \item Create new user
  \item Update user quota limits
  \item Delete user
\end{enumerate}



\subsection{BibTeX}

\noindent Read operations:
\begin{enumerate}
\item Get the full representation of a BibTeX entity
\item Search given a keyword
\item List all BibTeX entries
\end{enumerate}

\noindent Write operations:
\begin{enumerate}
 \item Create new BibTeX
 \item Update BibTeX
 \item Delete BibTeX
\end{enumerate}


\noindent Remarks:\\
BibTeX entities have really a lot of fields and it wouldn’t make sense 
to have indexes for every such field. In these cases, we can introduce simply 
a new keyword field that will be indexed and will do the job.


\subsection{Features}
\noindent Read operations
\begin{enumerate}
 \item Get representation of a feature by ID
 \item List features (not too important)
 \item Search features that are sameAs some other feature or feature category (ontology)
 \item Search for features by \texttt{hasSource}
 \item Search for features by their name or some keyword
\end{enumerate}
 

\noindent Write operations
\begin{enumerate}
 \item Create new feature
 \item Update a feature (replace)
 \item Delete feature
\end{enumerate}



\subsection{Compounds}

\noindent Read operations:
\begin{enumerate}
 \item List compound IDs (paginated)
\item  Fetch the representation of a compound ID
\item  Search for the ID of a compound: (i) according to a keyword (we can use ElasticSearch) 
       and/or (ii) the is \texttt{sameAs} some other feature or feature category from an ontology
\item  Fetch a compound along with some features [returns a dataset]. 
   The corresponding API method is \texttt{GET /compound/\{cid\}/feature}
\item  Get all the conformers of a given compound
\end{enumerate}


\noindent Write operations:
\begin{enumerate}
\item Write a new compound in the DB
\item  Update compound
\item  Delete compound
\item  Add conformers to a given compound
\item  Update the representation of a conformer (using PUT)
\end{enumerate}


\noindent Remarks
\begin{enumerate}
\item  It is important to decide whether we will be saving every 
representation of the compound in the database, or whether the new 
representation will be calculated on the fly from what is found in the DB. 
\textit{e.g.}, A client uploads the compound ‘aspirin’ on jaqpot4 in SDF and 
creates a new resource, namely \texttt{/compound/123}. 
Then one needs to retrieve \texttt{/compound/123} is MOL format. 
Will this be generated on the fly? I guess this depends on how fast 
this conversion can be done. 
\item When retrieving datasets from the DB we do not need to 
retrieve any representation of the compound. All compounds appear as links in the dataset.
\end{enumerate}



\subsection{FeatureValuePair}

The only operation that is related to FeatureValuePair entities is the one implied by the API method 
\texttt{POST /compound/\{cid\}/feature?feature\_uri=\{feature\}\& value=\{value\}} 
where a new value is stored in the DB as a FeatureValuePair

\subsection{Dataset}

\noindent Read operations
\begin{enumerate}
 \item Get all dataset IDs (list, paginated)
\item  Fetch a dataset of given ID
\item  List all datasets (IDs) that have a property (\textit{e.g.}, 
   \texttt{created by user x} or \texttt{are annotated as \#jaqpot}). This can be simply done using ElasticSearch!
\item  Get all features and feature values of a given compound (also discussed in Section “Compound”)
\item  Merge datasets
\item  Add compounds to datasets (creates new dataset on the fly)
\item  Fetch a dataset and append a few features (similar to merging)
\end{enumerate}


\noindent Write operations
\begin{enumerate}
\item  Create a new dataset (store all its values and the whole structure in the DB)
\item  Update a dataset (replace existing one)
\item Delete a dataset
\end{enumerate}

\noindent Remark: It seems we cannot store datasets in compact documents for many reasons. 
One reason is that whenever a feature-value pair entry is modified, this needs to be known to 
all datasets that use this value. \textit{e.g.}, if a client needs to modify the value of \texttt{molecular mass}
of \texttt{aspirin}, this needs to be updated in all datasets. 
Datasets are therefore collections of DataEntry entities.




\subsection{Algorithm}

There are certain cases that we need to store Algorithm entities
in the DB. Descriptor calculation algorithms (especially when they 
admit input parameters) may need to store new algorithm objects in the database. 
See http://opentox.org/
dev/apis/api-1.2/Algorithm\#section-7 for details. 
Algorithms are simply stored and retrieved as documents. Search is necessary only 
to return those algorithms that are sameAs some ontological category (need for one index). 
In detail we have:

\noindent Read operations
\begin{enumerate}
 \item List all algorithms
\item  Fetch algorithm by ID
\item  List algorithms that are sameAs a given ontological class
\item  List all algorithms generated by a user
\end{enumerate}

\noindent Write operations\\
Register a new algorithm

\subsection{Policy}

\noindent Read operations
\begin{enumerate}
 \item Fetch a policy of a given ID
\item  List all policies of a given user
\item  Search for a policy for a URI (allowed to the owner of the policy only)
\end{enumerate}


\noindent Write operations
\begin{enumerate}
\item  Register policy
\item  Update policy (replace)
\item  Delete policy
\end{enumerate}